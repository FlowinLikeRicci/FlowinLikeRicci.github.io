\documentclass[12pt]{book}
\usepackage{amsmath} % AMS Math Package
\usepackage{amsthm} % Theorem Formatting
\usepackage{amssymb}    % Math symbols such as \mathbb
\usepackage{graphicx} % Allows for eps images
\usepackage[dvips,letterpaper,margin=1in,bottom=0.7in]{geometry}
\usepackage{amsfonts,latexsym,amsthm,amssymb,amsmath,amscd,euscript}
\usepackage{framed}
\usepackage{enumitem}
\usepackage{algorithm}
\usepackage{algorithmicx}
\usepackage{algpseudocode}
\usepackage{fullpage}
\usepackage{hyperref}
    \hypersetup{colorlinks=true,citecolor=blue,urlcolor =black,linkbordercolor={1 0 0}}

\newenvironment{statement}[1]{\smallskip\noindent\color{black}{\bf #1.}}{}
\allowdisplaybreaks[1]
%Below are the theorem, definition, example, lemma, etc. body types.

\newtheorem{theorem}{Theorem}
\newtheorem*{proposition}{Proposition}
\newtheorem{lemma}[theorem]{Lemma}
\newtheorem{corollary}[theorem]{Corollary}
\newtheorem{conjecture}[theorem]{Conjecture}
\newtheorem{postulate}[theorem]{Postulate}
\theoremstyle{definition}
\newtheorem{defn}[theorem]{Definition}
\newtheorem{example}[theorem]{Example}

\theoremstyle{remark}
\newtheorem*{remark}{Remark}
\newtheorem*{notation}{Notation}
\newtheorem*{note}{Note}

% You can define new commands to make your life easier.
\newcommand{\BR}{\mathbb R}
\newcommand{\BC}{\mathbb C}
\newcommand{\BF}{\mathbb F}
\newcommand{\BQ}{\mathbb Q}
\newcommand{\BZ}{\mathbb Z}
\newcommand{\BN}{\mathbb N}
\newcommand{\Sq}{\mathrm{Sq}}
\newcommand{\mt}{\mathcal{T}}
\newcommand{\mo}{\mathcal{O}}
\newcommand{\p}{\partial}
\newcommand{\rank}{\mathrm{rank}}
\newcommand{\ms}{\mathcal{S}}
\newcommand{\id}{\mathrm{id}}
\newcommand{\seq}{\subseteq}
\newcommand{\s}{\subset}
\newcommand{\dint}{\displaystyle\int}
\DeclareRobustCommand{\rchi}{{\mathpalette\irchi\relax}}
\newcommand{\irchi}[2]{\raisebox{\depth}{$#1\chi$}}
% We can even define a new command for \newcommand!
\newcommand{\nc}{\newcommand}
\renewcommand\qedsymbol{$\blacksquare$}
\let\oldemptyset\emptyset
\let\emptyset\varnothing
\usepackage{scalerel}
\let\lctau\tau % save the lowercase of '\tau'
\newcommand{\overbar}[1]{\mkern 1.5mu\overline{\mkern-1.5mu#1\mkern-1.5mu}\mkern 1.5mu}
\renewcommand{\tau}{\scalerel*{\lctau}{X}}
\usepackage{times}
\usepackage{textcomp}
\begin{document}
\begin{center}
\textbf{\Large Construction of Stiefel-Whitney Classes}
\end{center}

In these notes, we will construct important topological invariants: Stiefel-Whitney classes. These cohomology classes classify vector bundles to some extent. Let $\pi : E \to B$ be a vector bundle. A straightforward way to define the Stiefel-Whitney classes is by the following axioms:
\begin{enumerate}
\item The $i$-th Stiefel-Whitney class, $w_{i}(E)$, is an element of $H^{i}(B;\BZ/2)$, $w_{0}(E) = 1$, and $w_{j}(E) = 0$ for $j > \rank(E).$
\item The Stiefel-Whitney classes are natural in the sense that they commute with pullbacks.
\item The total Stiefel-Whitney class, $w(E) = \sum w_{j}(E)$, of a direct sum splits as a cup product: $w(E_{1} \oplus E_{2}) = w(E_{1}) \smile w(E_{2}).$
\item The tautological line bundle over $\mathbb{RP}^{1}$ has non-zero $w_{1}.$
\end{enumerate}

This gives us some nice immediate properties, such as $w_{i}(E_{1} \oplus E_{2}) = w_{i}(E_{1})$, if $E_{2}$ is trivial. Another interesting property is that $w_{j}(E) = 0$ for $j > n - k$ if there are $k$ fiber-wise linearly independent non-vanishing sections of $E.$ One can show that, in fact, real vector bundles have Stiefel-Whitney classes described by a pullback of the class on the canonical bundle over the infinite Grassmannian, $\mathrm{G}_{n}(\BR^{\infty})$ of the corresponding dimension. Using these ideas, we can show that Stiefel-Whitney classes are unique.

Now, there exists a unique cohomology class, $u \in H^{n}(E,E_{0})$ that restricts to the nonzero element of $H^{n}(F,F_{0})$ in each fiber, where a zero subscript denotes a subtraction of the image of the zero section. Cup product with $u$ then yields an isomorphism, and additionally, $B \hookrightarrow E$ as the zero section, which means that $E$ deformation retracts onto $B$ and yields an isomorphism $\pi^{*} : H^{k}(B) \to H^{k}(E).$ Hence, the map $\phi : H^{k}(B) \to H^{n + k}(E,E_{0})$ is an isomorphism called the Thom isomorphism.

Another important element of this construction is the construction of the Steenrod squaring operations. These are written with respect to $\BZ/2$ coefficients and are characterized by the axioms:
\begin{enumerate}
\item For spaces, $A \s X$, and integers $i$ and $n$, there is a homomorphism $\Sq^{i} : H^{n}(X,A) \to H^{n + i}(X,A).$
\item These homomorphisms are natural in the same sense as above.
\item $\Sq^{0} = \id$, $\Sq^{n}$ is the cup square, and $\Sq^{i} = 0$ if $i > n$ by definition.
\item If $\Sq = \sum \Sq^{i}$, then $\Sq(a \smile b) = \Sq(a) \smile \Sq(b).$
\end{enumerate}
It follows that we can define $w_{i}(E) = \phi^{-1}\Sq^{i}\phi(1)$, which can be shown to coincide with the original definition.
\end{document}
\documentclass[12pt]{article}
\usepackage{amsmath} % AMS Math Package
\usepackage{amsthm} % Theorem Formatting
\usepackage{amssymb}    % Math symbols such as \mathbb
\usepackage{graphicx} % Allows for eps images
\usepackage[dvips,letterpaper,margin=1in,bottom=0.7in]{geometry}
\usepackage{amsfonts,latexsym,amsthm,amssymb,amsmath,amscd,euscript}
\usepackage{framed}
\usepackage{enumitem}
\usepackage{algorithm}
\usepackage{algorithmicx}
\usepackage{algpseudocode}
\usepackage{fullpage}
\usepackage{hyperref}
    \hypersetup{colorlinks=true,citecolor=blue,urlcolor =black,linkbordercolor={1 0 0}}

\newenvironment{statement}[1]{\smallskip\noindent\color{black}{\bf #1.}}{}
\allowdisplaybreaks[1]
%Below are the theorem, definition, example, lemma, etc. body types.

\newtheorem{theorem}{Theorem}
\newtheorem*{proposition}{Proposition}
\newtheorem{lemma}[theorem]{Lemma}
\newtheorem{corollary}[theorem]{Corollary}
\newtheorem{conjecture}[theorem]{Conjecture}
\newtheorem{postulate}[theorem]{Postulate}
\theoremstyle{definition}
\newtheorem{defn}[theorem]{Definition}
\newtheorem{example}[theorem]{Example}

\theoremstyle{remark}
\newtheorem*{remark}{Remark}
\newtheorem*{notation}{Notation}
\newtheorem*{note}{Note}

% You can define new commands to make your life easier.
\newcommand{\BR}{\mathbb R}
\newcommand{\BC}{\mathbb C}
\newcommand{\BF}{\mathbb F}
\newcommand{\BQ}{\mathbb Q}
\newcommand{\BZ}{\mathbb Z}
\newcommand{\BN}{\mathbb N}
\newcommand{\mt}{\mathcal{T}}
\newcommand{\mo}{\mathcal{O}}
\newcommand{\p}{\partial}
\newcommand{\ms}{\mathcal{S}}
\newcommand{\seq}{\subseteq}
\newcommand{\s}{\subset}
\newcommand{\dint}{\displaystyle\int}
\DeclareRobustCommand{\rchi}{{\mathpalette\irchi\relax}}
\newcommand{\irchi}[2]{\raisebox{\depth}{$#1\chi$}}
% We can even define a new command for \newcommand!
\newcommand{\nc}{\newcommand}
\renewcommand\qedsymbol{$\blacksquare$}
\let\oldemptyset\emptyset
\let\emptyset\varnothing
\usepackage{scalerel}
\let\lctau\tau % save the lowercase of '\tau'
\newcommand{\overbar}[1]{\mkern 1.5mu\overline{\mkern-1.5mu#1\mkern-1.5mu}\mkern 1.5mu}
\renewcommand{\tau}{\scalerel*{\lctau}{X}}
\usepackage{times}
\usepackage{textcomp}
\newcommand{\bigslant}[2]{{\raisebox{.2em}{$#1$}\left/\raisebox{-.2em}{$#2$}\right.}}
\begin{document}
{\noindent\Huge\bf  \\[0.5\baselineskip] {\fontfamily{cmr}\selectfont  Selected Exercises}         }\\[1\baselineskip] % Title
{ {\bf \fontfamily{cmr}\selectfont Complex Analysis}\\ {\textit{\fontfamily{cmr}\selectfont     \today}}}~~~~~~~~~~~~~~~~~~~~~~~~~~~~~~~~~~~~~~~~~~~~~~~~~~~~~~~~~~~~~~~~~~~~~~~~~~~~~    {\large{Jacob Krantz}}
\noindent\makebox[\linewidth]{\rule{8in}{0.4pt}}
\\[1.4\baselineskip]
Here are selected problems I solved while self-studying \textit{Complex Analysis} by Stein and Sharkarchi.
\begin{statement}{1.9}
We first deduce the Cauchy-Riemann equations in polar coordinates. Notice that:
\begin{align*}
\dfrac{\p u}{\p r} &= \dfrac{\p u}{\p x}\dfrac{\p x}{\p r} + \dfrac{\p u}{\p y}\dfrac{\p y}{\p r} \\\\
&= \dfrac{\p u}{\p x}\cos\theta + \dfrac{\p u}{\p y}\sin\theta \\\\
&= \dfrac{1}{r}\dfrac{\p v}{\p y}r\cos\theta - \dfrac{1}{r}\dfrac{\p v}{\p x}r\sin\theta \\\\
&= \dfrac{1}{r}\dfrac{\p v}{\p \theta} \\\\
\dfrac{1}{r}\dfrac{\p u}{\p \theta} &= \dfrac{1}{r}\left (\dfrac{\p u}{\p x}\dfrac{\p x}{\p \theta} + \dfrac{\p u}{\p y}\dfrac{\p y}{\p \theta}\right ) \\\\
&= \dfrac{1}{r}\left (-\dfrac{\p u}{\p x}r\sin\theta + \dfrac{\p u}{\p y}r\cos\theta\right ) \\\\
&= -\dfrac{\p v}{\p y}\sin\theta - \dfrac{\p v}{\p x}\cos\theta \\\\
&= -\dfrac{\p v}{\p r}.
\end{align*}
Write $\log z = \log r + i\theta = u(r,\theta) + iv(r,\theta).$ By Theorem $2.4,$ it suffices to show $u$ and $v$ satisfy the Cauchy-Riemann equations. Clearly, for the given domain, we have
\[\dfrac{\p u}{\p r} = \dfrac{1}{r} \quad \dfrac{\p u}{\p \theta} = 0 \quad \dfrac{\p v}{\p r} = 0 \quad \dfrac{\p v}{\p \theta} = 1.\]
Thus, we get that
\[\dfrac{\p u}{\p r} = \dfrac{1}{r} = \dfrac{1}{r}\dfrac{\p v}{\p \theta}\]
and
\[\dfrac{1}{r}\dfrac{\p u}{\p \theta} = 0 = -\dfrac{\p v}{\p r}.\]
Thus, $\log z$ is holomorphic for $r > 0$ and $\theta \in (-\pi,\pi).$
\end{statement}
\\\\
\begin{statement}{1.10}
Applying the differentiation operators as usual, we compute
\begin{align*}
4\dfrac{\p}{\p z}\dfrac{\p}{\p\overline{z}} &= 2\dfrac{\p}{\p z}\dfrac{\p}{\p x} - \dfrac{2}{i}\dfrac{\p}{\p z}\dfrac{\p}{\p y} \\\\
&= \dfrac{\p^{2}}{\p x^{2}} + \dfrac{1}{i}\dfrac{\p^{2}}{\p y\p x} - \dfrac{1}{i}\dfrac{\p^{2}}{\p x\p y} + \dfrac{\p^{2}}{\p y^{2}} \\\\
&= \Delta \\\\
4\dfrac{\p}{\p \overline{z}}\dfrac{\p}{\p z} &= 2\dfrac{\p}{\p \overline{z}}\dfrac{\p}{\p x} + \dfrac{2}{i}\dfrac{\p}{\p \overline{z}}\dfrac{\p}{\p y} \\\\
&= \dfrac{\p^{2}}{\p x^{2}} - \dfrac{1}{i}\dfrac{\p^{2}}{\p y\p x} + \dfrac{1}{i}\dfrac{\p^{2}}{\p x\p y} + \dfrac{\p^{2}}{\p y^{2}} \\\\
&= \Delta.
\end{align*}
\end{statement}
\begin{statement}{1.24}
Using the reverse parameterization defined in the book, we observe
\begin{align*}
\int_{\gamma^{-}}f(x)\;dz &= \int_{a}^{b}f(z^{-}(t))(z^{-})'(t)\;dt \\\\
&= -\int_{a}^{b}f(z(a + b - t))z'(b + a - t)\;dt \\\\
&= \int_{b}^{a}f(z(u))z'(u)\;du \\\\
&= -\int_{a}^{b}f(z(u))z'(u)\;du \\\\
&= -\int_{\gamma}f(z)\;dz.
\end{align*}
\end{statement}
\end{document}